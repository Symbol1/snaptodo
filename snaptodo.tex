
\documentclass{article}

\usepackage{snaptodo}

\usepackage[pdfusetitle]{hyperref}

\begin{document}

\title{Snaptodo---Snap to the Correct Side Todo}

\author{Hsin-Po Wang\footnote{A.Simple.People@gmail.com}}

\maketitle

\begin{abstract}
	Snaptodo package is an alternative to the todonote package
	that differs in the following ways:
	(A) Depending on where you call \texttt{\string\todo}, the note
		is put on the left or right margin, whichever is closer.
	(B) The notes bump each other down so they never overlap;
		the lines never overlap either; and they don't
		trigger the \texttt{\string\marginpar} warning.
	(C) Minimalistic, aesthetic, and customizable style.

\end{abstract}

\section{Minimal Working Example}

	\begin{verbatim}
		\documentclass{article}
		\usepackage{snaptodo}
		\begin{document}
		    Your done
		    \todo{Your todo}
		\end{document}
	\end{verbatim}

\section{Examples Showcase}

	Using snaptodo is as simple as calling
	{\ttfamily\verb|\todo|}%
	\todo{Calling \texttt{\string\todo}!}
	and compile twice.%
	\todo{Compile twice!}
	If the calling point%
	\todo{Calling point!}
	is on the left of a page,%
	\todo{Left of a page!}
	the todo note, after compiling twice of course,
	will appear to the left margin.%
	\todo{On the left margin!}
	And vice versa.%  
	\todo{Vice versa!}

\section{Customization}

	I defined
	{\ttfamily\verb|\snaptodoset|}
	that works like
	{\ttfamily\verb|\tikzset|} and
	{\ttfamily\verb|\pgfplotsset|}.
	While you can access some options by
	{\ttfamily\verb|\snaptodoset{_some_option_}|},
	you can also assess the exact same options by
	{\ttfamily\verb|\pgfkeys{/snaptodo/_some_option_}|}.

	The color of the broken line is
	{\ttfamily\verb|snaptodo@chain|}.
	The default color for that is
	{\ttfamily\verb|red!50!black|}.
	Using
	{\ttfamily\verb|\colorlet{snaptodo@chain}{green!50!white}|}
	lets you to change this color globally.
	Instead, for locally changing this color, one may prefer using
	{\ttfamily\verb|\todo[call chain/.append style=green!50!white]|}%
	\todo[call chain/.append style=green!50!white]{New broken line color}.

	The color of the note text is
	{\ttfamily\verb|snaptodo@block|}.
	The default color for that is
	{\ttfamily\verb|yellow!50!black|}.
	Using
	{\ttfamily\verb|\colorlet{snaptodo@block}{blue!50!white}|}
	lets you change this color.
	Instead, for locally changing this color, one may prefer using
	{\ttfamily\verb|\todo[margin block/.append style=blue!50!white]|}%
	\todo[margin block/.append style=blue!50!white]{New note text color}.

	The gap between two notes is stored in
	{\ttfamily\verb|/snaptodo/block sep|}.
	The default length for that is
	{\ttfamily\verb|\baselineskip|}.
	To change the behavior, you may use
	{\ttfamily\verb|\snaptodoset{block sep=0pt}|}
	for global and
	{\ttfamily\verb|\todo[block sep=0pt]{}|}%
	\todo[block sep=0pt]{2 blocks close.}
	\todo[block sep=0pt]{..to each other}
	for local setting.
	
	The gap between two lines is stored in
	{\ttfamily\verb|/snaptodo/chain sep|}.
	The default length for that is
	{\ttfamily\verb|0.5ex|}.
	To change, use
	{\ttfamily\verb|\snaptodoset{chain sep=0pt}|}
	for global and
	{\ttfamily\verb|\todo[chain sep=0pt]{}|}%
	\todo[chain sep=0pt]{2 chains close.}
	\todo[chain sep=0pt]{..to each other}
	for local setting.
	
	The alignment of the text in the note
	can be controlled by the choices:
	{\ttfamily\verb|far|},%
	\todo[far]{far}
	which is left on the left and right on the right,
	{\ttfamily\verb|flush far|},%
	\todo[flush far]{flush far}
	which is flush left on the left and flush right on the right,
	{\ttfamily\verb|near|},%
	\todo[near]{near}
	which is right on the left and left on the right,
	{\ttfamily\verb|flush near|},%
	\todo[near]{flush near}
	which is flush right on the left and flush left on the right,
	{\ttfamily\verb|center|},%
	\todo[center]{center}
	which is center on both sides.
	{\ttfamily\verb|flush center|},%
	\todo[flush center]{flush center}
	which is flush center on both sides.

\end{document}