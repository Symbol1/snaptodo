
\documentclass{article}

\def\snaptodo{}
\usepackage{snaptodo}

\usepackage[pdfusetitle,urlbordercolor=red!50!black]{hyperref}

\begin{document}

\title{Snaptodo---A Snap-to-the-Correct-Side Todo\thanks{
	Repository
	\url{https://github.com/Symbol1/snaptodo}.
}}

\author{Hsin-Po Wang\thanks{
	Email
	\href{mailto:a.simple.people@gmail.com}{\ttfamily a.simple.people@gmail.com}.
	Website
	\url{https://www.symbol.codes}.}}

\date{December 1, 2021}

\maketitle

\begin{abstract}
	Snaptodo package is an alternative to the todonote package
	that differs in the following ways:
	(A)	Depending on where you call
		{\ttfamily\color{red!50!black}\string\snaptodo},
		the note is put on the left or right margin,
		whichever is closer.
	(B)	The notes bump each other down so they never overlap;
		the lines never overlap either;
		and they don't trigger the marginpar warning.
	(C)	Minimalistic, aesthetic, and customizable style.

\end{abstract}

\section{Installation}

	Copy and paste
	{\ttfamily\verb|snaptodo.sty|}
	to your working directory.

\section{Minimal Working Example}

	\begin{verbatim}
		\documentclass{article}
		\usepackage{snaptodo}
		\begin{document}
		    Your done
		    \snaptodo{Your todo}
		\end{document}
	\end{verbatim}

\section{Showcase of Examples}

	Using snaptodo is as simple as calling
	{\ttfamily\verb|\snaptodo|}%
	\snaptodo{Calling \texttt{\string\snaptodo}!}
	and compile twice.%
	\snaptodo{Compile twice!}
	If the calling point%
	\snaptodo{Calling point!}
	is on the left of a page,%
	\snaptodo{Left of a page!}
	the todo note---after compiling twice of
	course---will appear to the left margin.%
	\snaptodo{On the left margin!}
	And vice versa.%  
	\snaptodo{Vice versa!}

\section{Customization}

	I defined
	{\ttfamily\verb|\snaptodoset|}
	that works like
	{\ttfamily\verb|\tikzset|} and
	{\ttfamily\verb|\pgfplotsset|}.
	While you can access dedicated options by
	{\ttfamily\verb|\snaptodoset{<some_option>}|},
	you can also assess the exact same options by
	{\ttfamily\verb|\pgfkeys{/snaptodo/<some_option>}|}.
	For local changes, you can put them here
	{\ttfamily\verb|\snaptodo[<some_option>]{}|}.

\subsection{Color}

	The color of the broken line is
	{\ttfamily\verb|snaptodo@chain|}.
	The default color for that is
	{\ttfamily\verb|red!50!black|}.
	Saying
	{\ttfamily\verb|\colorlet{snaptodo@chain}{green!50!white}|}
	lets you to change this color globally.  On the other hand,
	for locally changing this color, one may prefer using
	{\ttfamily\verb|\snaptodo[call chain/.style=green!50!white]|}.%
	\snaptodo[call chain/.style=green!50!white]{New broken line color}

	The color of the note text is
	{\ttfamily\verb|snaptodo@block|}.
	The default color for that is
	{\ttfamily\verb|yellow!50!black|}.
	Saying
	{\ttfamily\verb|\colorlet{snaptodo@block}{blue!50!white}|}
	lets you change this color.  On the other hand,
	for locally changing this color, one may prefer using
	{\ttfamily\verb|\snaptodo[margin block/.style=blue!50!white]|}.%
	\snaptodo[margin block/.style=blue!50!white]{New note text color}

\subsection{Line style}

	In order to use a thicker or thinner broken line,
	recall the standard Ti\emph kZ option
	{\ttfamily\verb|[thick]|},
	{\ttfamily\verb|[ultra thin]|},
	or
	{\ttfamily\verb|[line width=???]|}.
	In our case, simply put that option in this style
	{\ttfamily\verb|\snaptodo[call chain/.style={line width=???}]{}|}.%
	\snaptodo[call chain/.style={line width=2pt}]{Thicker broken line}

	In order to use a special dash pattern,
	recall the standard Ti\emph kZ option
	{\ttfamily\verb|[dotted]|},
	{\ttfamily\verb|[dashed]|},
	or
	{\ttfamily\verb|[dash pattern=???]|}.
	In our case, simply put that option in this style
	{\ttfamily\verb|\snaptodo[call chain/.style={dash pattern=???}]{}|}.%
	\snaptodo[call chain/.style={dashed}]{Dashed broken line}

\subsection{Sep and width}

	The gap between two note blocks is stored in
	{\ttfamily\verb|/snaptodo/block sep|}.
	The default length is
	{\ttfamily\verb|\baselineskip|}.
	Use
	{\ttfamily\verb|\snaptodoset{block sep=0pt}|}
	for global setting and use
	{\ttfamily\verb|\snaptodo[block sep=0pt]{note}|}%
	\snaptodo[block sep=0pt]{2 blocks close.}
	\snaptodo[block sep=0pt]{..to each other}
	for local setting this.
	
	The gap between two horizontal lines is stored in
	{\ttfamily\verb|/snaptodo/chain sep|}.
	The default length is
	{\ttfamily\verb|0.5ex|}.
	To change, use
	{\ttfamily\verb|\snaptodoset{chain sep=0pt}|}
	for global setting and
	{\ttfamily\verb|\snaptodo[chain sep=0pt]{note}|}%
	\snaptodo[chain sep=0pt]{2 chains close.}
	\snaptodo[chain sep=0pt]{..to each other}
	for local setting.

	The width of the note block is
	{\ttfamily\verb|\marginparwidth|}.
	The width where the broken line has slope is
	{\ttfamily\verb|\marginparsep|}.
	These are the built-in dimensions.
	(So modify them with care!)

\subsection{Bias}

	The tipping point between snapping to the left versus
	to the right is the center of the page, by default.
	You can change this by
	{\ttfamily\verb|/snaptodo/chain bias=-99in|}.%
	\snaptodo[chain bias=-99in]{Forced to the left}
	That way, all todo notes are forcedly
	snapped to the left hand side.  Or, by
	\snaptodo[chain bias=99in]{Forced to the right}%
	{\ttfamily\verb|/snaptodo/call bias=99in|},
	all todo notes are forcedly snapped to the right hand side.

\subsection{Rise}

	If your document is really busy, e.g., like
	{\ttfamily\verb|stress_testing.tex|},
	consider setting a positive
	{\ttfamily\verb|/snaptodo/block rise|}.%
	\snaptodo[block rise=2em]{Rising}
	\snaptodo[block rise=2em]{to leave}
	\snaptodo[block rise=2em]{room.}
	For instance,
	{\ttfamily\verb|\snaptodo[block rise=2em]{}|}.
	That way, a note will rise by that amount
	to leave more rooms for later notes.
	(But if there will be overlay,
	the notes automatically bump down.)


\subsection{Alignment}

	By
	{\ttfamily\verb|/snaptodo/margin block/.style={align=???}|}
	one can control the alignment of note text.
	The default setting is
	{\ttfamily\verb|align=flush \std@leftright|},
	which flushes the text toward the page edges.
	If
	{\ttfamily\verb|align=flush \std@rightleft|},
	is what you do,%
	\makeatletter
	\snaptodo[margin block/.style={align=flush \std@rightleft}]
	{flush toward text body}
	\makeatother
	the text will be flushed toward the main
	text body.  Don't forget that you need
	{\ttfamily\verb|\makeatletter|}
	and
	{\ttfamily\verb|\makeatother|}
	to handel control sequences with the at character.

\subsection{Font}

	Font size is controlled by
	\snaptodo[margin block/.style={font=\tiny}]{tiny}
	\snaptodo[margin block/.style={font=\Large}]{Large}
	\snaptodo[margin block/.style={font=\Huge}]{Huge}
	{\ttfamily\verb|/snpatodo/margin block/.style={font=...}|}.
	\snaptodo[margin block/.style={font=\itshape}]{itshape}
	\snaptodo[margin block/.style={font=\sffamily}]{sffamily}
	\snaptodo[margin block/.style={font=\bfseries}]{bfseries}
	So is font family.

\end{document}
